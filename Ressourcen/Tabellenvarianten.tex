\addchap{Tabellenvarianten}

\vspace{22mm}
\section*{Überschrift Tabelle 1}

\begin{table}[!h]
\begin{tabularx}{\textwidth + 5pt}{@{\hspace{3pt}} M | @{\hspace{3pt}} M}
\multicolumn{2}{@{}X}{%
    \begin{tabularx}{\textwidth}{@{\hspace{3pt}} M @{\hspace{14.5pt}} M}
    \textbf{Spalte 1} & \textbf{Spalte 2}
    \end{tabularx}%
} \\
\hline
Nummer 1 & Nummer 2 \\
\hline
Nummer 1 & Nummer 2 \\
\hline
Nummer 1 & Nummer 2 \\
\hline
\end{tabularx}

\caption{Beschreibung}
\end{table}


\vspace{\parskip}
\section*{Überschrift Tabelle 2}

\begin{table}[!h]
\hspace{-5pt}
\begin{tabularx}{\textwidth + 5pt}{| @{\hspace{3pt}} M | @{\hspace{3pt}} M |}
\hline
\textbf{Spalte 1} & \textbf{Spalte 2} \\
\hline
Nummer 1 & Nummer 2 \\
\hline
Nummer 1 & Nummer 2 \\
\hline
Nummer 1 & Nummer 2 \\
\hline
\end{tabularx}
\caption{Beschreibung}
\end{table}


\vspace{\parskip}
\section*{Überschrift Tabelle 3}

\begin{table}[!h]
\begin{tabularx}{\textwidth}{@{} M M}
\textbf{Spalte 1} & \textbf{Spalte 2} \\
Nummer 1 & Nummer 2 \\
Nummer 1 & Nummer 2 \\
Nummer 1 & Nummer 2 \\
\end{tabularx}
\caption{Beschreibung}
\end{table}

\clearpage

\addchap{Tabellenvarianten 2}

\vspace{22mm}
\section*{Überschrift Tabelle 1}

\begin{table}[!h]
\fontsize{9pt}{13pt}\selectfont
%\renewcommand{\arraystretch}{1.8}
\hspace{-5pt}
\begin{tabularx}{\textwidth + 5pt}{@{\hspace{3pt}} M | @{\hspace{3pt}} M}
\multicolumn{2}{@{}X}{%
    \begin{tabularx}{\textwidth}{@{\hspace{3pt}} M @{\hspace{14.5pt}} M}
    \textbf{Spalte 1} & \textbf{Spalte 2}
    \end{tabularx}%
} \\
\hline
Nummer 1,\newline\,mehrzeilig in Schriftgröße 9 pt & Nummer 2 \\
\hline
Nummer 1 & Nummer 2 \\
\hline
Nummer 1 & Nummer 2 \\
\hline
\end{tabularx}

\caption{Beschreibung}
\end{table}


\vspace{\parskip}
\section*{Überschrift Tabelle 2}

\begin{table}[!h]
\fontsize{9pt}{13pt}\selectfont
\hspace{-5pt}
%\renewcommand{\arraystretch}{1.8}
\begin{tabularx}{\textwidth + 5pt}{| @{\hspace{3pt}} M | @{\hspace{3pt}} M |}
\hline
\textbf{Spalte 1} & \textbf{Spalte 2} \\
\hline
Nummer 1 & Nummer 2 \\
\hline
Nummer 1 & Nummer 2 \\
\hline
Nummer 1 & Nummer 2 \\
\hline
\end{tabularx}
\caption{Beschreibung}
\end{table}


\vspace{\parskip}
\section*{Überschrift Tabelle 3}

\begin{table}[!h]
\fontsize{9pt}{13pt}\selectfont
%\renewcommand{\arraystretch}{1.8}
\begin{tabularx}{\textwidth}{@{} M M}
\textbf{Spalte 1} & \textbf{Spalte 2} \\
Nummer 1 & Nummer 2 \\
Nummer 1 & Nummer 2 \\
Nummer 1 & Nummer 2 \\
\end{tabularx}
\caption{Beschreibung}
\end{table}