%%%%%%%%%%%%%%%%%%%%%%%%%%%%%%%%%%%%%%%%%%%%%%%%%%%%%%%%
%%%%                                              %%%%%%
%%%%  Author: Peter Wilson                        %%%%%%
%%%%                                              %%%%%%
%%%%  Derivation of Euler buckling load                        %%%%%%
%%%%                                              %%%%%%
%%%%%%%%%%%%%%%%%%%%%%%%%%%%%%%%%%%%%%%%%%%%%%%%%%%%%%%%
\chapter[Derivation of Euler buckling load]{Derivation of Euler\\ buckling load}
\label{app:Derivation of Euler buckling load}
\renewcommand{\Thema}{Derivation of Euler buckling load}

As part of the stability analysis of a CHS beam considered in section \ref{applications: Euler buckling of CHS column} an Euler buckling solution is included for reference, the derivation of which is presented here.

The governing differential for a CHS section beam with Young's Modulus $E$  and a second moment of area $I_{xx} = I_{yy} = I$ subject solely to an axial compressive load $P$ without any other loading or spring beds imposed is:

\begin{equation} 
EI w'''' + Pw'' = 0
\label{eqapp4_1}\ .
\end{equation}

The general solution of the above 4th order differential is a displacement field of the form $w = Asin(\lambda x) + Bcos(\lambda x) +  Cx + D$ where $\lambda^2 = P/EI$. The arbitrary constants $A$ through $D$ can be determined by considering four boundary conditions of the beam which is fixed at both ends. This fixity arrangement prescribes the following displacements and angular deflections at both ends:

\begin{equation} 
w(0) = 0
\label{eqapp4_2}\ ,
\end{equation}
\begin{equation} 
w(L) = 0
\label{eqapp4_3}\ ,
\end{equation}
\begin{equation} 
w'(0) = 0
\label{eqapp4_4}
\end{equation}
and
\begin{equation} 
w'(L) = 0
\label{eqapp4_5}\ .
\end{equation}

These boundary conditions can be substituted into the general displacement field and then arranged in a matrix:

\begin{equation} 
\begin{pmatrix}
0 & 1 & 0 & 1 \\
sin(\lambda L) & cos(\lambda L) & L & 1\\
\lambda & 0 & 1 & 0 \\
\lambda cos(\lambda L) & -\lambda sin(\lambda L) & 1 & 0
\end{pmatrix}
\begin{pmatrix}
A \\
B \\
C \\
D
\end{pmatrix}
=
\mathbf{0}
\label{eqapp4_6}\ .
\end{equation}

The critical loads of the system precipitate from the eigenvalues of the matrix above. Thus, the determinant of the above matrix can be simplified and set to zero:

\begin{equation} 
\lambda L sin(\lambda L) + 2[cos(\lambda L) - 1] = 0
\label{eqapp4_7}\ .
\end{equation}

The bracketed term can be fulfilled with the following value of $\lambda$, which also satisfies the first term too:

\begin{equation} 
\lambda = \frac{2n\pi}{L}\ ,
(n = 1, 2, 3, ...)
\label{eqapp4_8}\ .
\end{equation}

Recalling the original definition of $\lambda$, one can combine the eigenvalue and express the critical buckling load of the beam as:

\begin{equation} 
P_{crit} = \frac{4\pi^2 n^2 EI}{L^2}
\label{eqapp4_9}\ .
\end{equation}

It's clear that the critical load can be minimized by accepting the first eigenvalue $n=1$, thus:

\begin{equation} 
P_{crit} = \frac{4\pi^2 EI}{L^2}
\label{eqapp4_10}\ .
\end{equation}