%%%%%%%%%%%%%%%%%%%%%%%%%%%%%%%%%%%%%%%%%%%%%%%%%%%%%%%%
%%%%                                              %%%%%%
%%%%  Author: Peter Wilson                        %%%%%%
%%%%                                              %%%%%%
%%%%  Stabilty analysis of Mises truss                        %%%%%%
%%%%                                              %%%%%%
%%%%%%%%%%%%%%%%%%%%%%%%%%%%%%%%%%%%%%%%%%%%%%%%%%%%%%%%

\chapter[Analytical stability analysis of Mises truss]{Analytical stability\\ analysis of Mises truss}
\label{app:Analytical stability analysis of Mises truss}

An analytical stability analysis of the Mises truss system considered in chapter \ref{chap:chapter_2_2} is presented employing the principle of virtual work via the 2nd Piola-Kirchhoff (PK2) stress measure and Green-Lagrange (GL) strain measure.

<Picture of truss>

The kinematics of the system are considered first by describing the undeformed truss lengths $L$, deformed lengths $l$ and the Green-Lagrange strain $\epsilon_{gl}$.

\begin{equation} 
L^2 = a^2 + b^2
\hspace{10mm}
l^2 = a^2 + (b-u)^2 = L^2 + u^2 - 2bu
\label{eqapp0_1}
\end{equation}

The Green-Lagrange strain measure for a truss is recalled and specified, as is it's first variation:

\begin{equation} 
\epsilon_{gl} = \frac{1}{2}
\Big(\frac{l^2-L^2}{L^2}\Big)
=
\Big(\frac{u^2 - 2bu}{2L^2}\Big)
\hspace{10mm}
\delta\epsilon_{gl} = 
\Big(\frac{u-b}{L^2}\Big)
\delta u
\label{eqapp0_2}
\end{equation}

Shifting towards stresses, the 2nd Piola-Kirchhoff stresses are linked to the Green-Lagrange strains via a linearly elastic constitutive law characterised by an axial Young's Modulus $E$.

\begin{equation} 
\sigma_{pk2} = E \epsilon_{gl}
\label{eqapp0_3}
\end{equation}

With the conjugate energy quantities defined, the virtual work expression of the system can be established:

\begin{equation} 
-\delta W = \delta W_int - \delta W_ext = 0
\label{eqapp0_4}
\end{equation}

Clarifying with respect to the system considered yields:

\begin{equation} 
\delta W_{int} - \delta W_{ext} = 
2 \int_{V} \sigma_{pk2} \epsilon_{gl}
\ dV
- \lambda P \delta u = 0
\label{eqapp0_5}
\end{equation}

Development of the above expression leads to the system residual $\mathbf{r}$.
\begin{gather} 
	\begin{aligned}
		&2 \int_{V} E \epsilon_{gl} \delta \epsilon_{gl}
		\ dV
		- \lambda P \delta u = 0		
		\\
		& 2EAL \Big(\frac{u^2 - 2bu}{2L^2}\Big) \Big(\frac{u-b}{L^2}\Big)
		\delta u - \lambda P \delta u = 0	
		\\
		&\mathbf{r} = 
		\frac{EA}{L^3}
		\Big[
		u^3 -3bu^2 +2b^2u
		\Big]
		- \lambda P
		\label{eqapp0_6}
	\end{aligned}
\end{gather}

Recalling that the system tangent matrix $\mathbf{K}$ is the gradient of the system residual, it can be expressed as follows:
\begin{gather} 
	\begin{aligned}
		&\mathbf{K} = 
		\frac{\partial \mathbf{r}}{\partial \mathbf{u}}
		\\
		&
		\mathbf{K} = 
		\frac{EA}{L^3}
		\Big[
		3u^2 -6bu +2b^2
		\Big]
		\label{eqapp0_7}
	\end{aligned}
\end{gather}

Key to stability analysis are the calculation of critical points which indicate the onset of instability. These points occur when the determinant of the tangent matrix vanishes, thus:

\begin{equation} 
\delta W_{int} - \delta W_{ext} = 
2 \int_{V} \sigma_{pk2} \epsilon_{gl}
\ dV
- \lambda P \delta u = 0
\label{eqapp0_8}
\end{equation}