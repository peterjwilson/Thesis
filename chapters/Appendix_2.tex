%%%%%%%%%%%%%%%%%%%%%%%%%%%%%%%%%%%%%%%%%%%%%%%%%%%%%%%%
%%%%                                              %%%%%%
%%%%  Author: Peter Wilson                        %%%%%%
%%%%                                              %%%%%%
%%%%  Background of shells                        %%%%%%
%%%%                                              %%%%%%
%%%%%%%%%%%%%%%%%%%%%%%%%%%%%%%%%%%%%%%%%%%%%%%%%%%%%%%%

\chapter{Basic-T3 formulation}
\label{sec:Basic-T3 quadrilateral formulation}
\renewcommand{\Thema}{Basic-T3 quadrilateral formulation}

Introduced in section \ref{section:shell obstacle course}, the Basic-T3 quadrilateral element corresponds to a basic constant strain triangle element without any DSG enhancements. It shares the same membrane and bending strain displacement matrices as the DSG triangle element (refer equations (\ref{eqt6}) and (\ref{eqt8})), but has a different transverse shear make-up and no correction of the material matrix (as per equation (\ref{eqt14})). 

The transverse shear stiffness of the Basic-T3 formulation is purely displacement based and employs the standard linear triangle shape functions as per the DSG element, repeated here:
\begin{gather} 
\begin{aligned}
&N_1 (x , y) = \frac{1}{2 A} \big[ (x_2 y_3 - x_3 y_2) + x(y_2 - y_3) + y(x_3 - x_2) \big]\ ,
\\
&N_2 (x , y) = \frac{1}{2 A} \big[ (x_3 y_1 - x_1 y_3) + x(y_3 - y_1) + y(x_1 - x_3) \big]\  \text{and}
\\
&N_3 (x , y) = \frac{1}{2 A} \big[ (x_1 y_2 - x_2 y_1) + x(y_1 - y_2) + y(x_2 - x_1) \big]
\label{eqApp2_1}\ .
\end{aligned}
\end{gather}
The transverse shear strains are related to the derivative of the discrete transverse displacements $v_{zi}$ and the value of nodal rotations $\beta_{xi}$ and $\beta_{yi}$ via the following transverse shear strain displacement matrix arrangement:
\begin{equation}
\boldsymbol{\gamma} = (\nabla \mathbf{N}^{v_{zi}} + \mathbf{N}^{\beta} ) \hat{\mathbf{v}} 
= \mathbf{B} \hat{\mathbf{v}}  = 
\begin{pmatrix}
\mathbf{B}_1 & \mathbf{B}_2 & \mathbf{B}_3 
\end{pmatrix}
\begin{pmatrix}
\hat{\mathbf{v}}_1 \\
 \hat{\mathbf{v}}_2 \\
  \hat{\mathbf{v}}_3 
\end{pmatrix}
\label{eqApp2_2}\ .
\end{equation}
The entries of $\mathbf{B}_i$ and $\hat{\mathbf{v}}_i $ are clarified:
\begin{equation}
\mathbf{B}_i=  \begin{pmatrix}
{N}_{i,x} & {N}_i & 0\\
{N}_{i,y} & 0 & {N}_i 
\end{pmatrix}
\hspace{5mm}
and
\hspace{5mm}
\hat{\mathbf{v}}_i =  \begin{pmatrix}
\hat{v}_{zi}\\
\hat{\beta}_{xi}\\
\hat{\beta}_{yi}
\end{pmatrix}
\label{eqApp2_3}\ .
\end{equation}