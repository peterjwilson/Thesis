%%%%%%%%%%%%%%%%%%%%%%%%%%%%%%%%%%%%%%%%%%%%%%%%%%%%%%%%
%%%%                                              %%%%%%
%%%%  Author: Peter Wilson                        %%%%%%
%%%%                                              %%%%%%
%%%%  Shell applications                       %%%%%%
%%%%                                              %%%%%%
%%%%%%%%%%%%%%%%%%%%%%%%%%%%%%%%%%%%%%%%%%%%%%%%%%%%%%%%


%fref generates automatically the respective abreviation/word in the text for the reference. You just have to define a label starting with the respective keyword.
%english: chap, sec, fig, eq, app
%deutsch: chap/kap, abs, abb, gl, anh
%see http://ctan.space-pro.be/tex-archive/macros/latex/contrib/fancyref/fancyref.pdf for more \section

%\onehalfspacing
%\setlength{\belowcaptionskip}{-17pt}

\chapter{Application of shell finite elements}
\label{chap:chapter_application}

\renewcommand{\Thema}{Application of shell finite elements}

\lettrine[lines=2]{T}{he} employment of shell structures is ubiquitous throughout both nature and the built environment. Eggs, nuts, blood vessels and cell walls are examples of shell designs being the result of structural optimisation via natural evolution over millennia. It is no doubt t

- Stability worksheet 6. Theory (full derivation) vs NL vs LPB
- Buckling of cylinder (TOS last lecture)

\section{Stability analysis}

asdfaf

\subsection{Euler buckling of CHS column}
\label{applications: Euler buckling of CHS column}
asdaf

\subsection{Shear wrinkling of thin membrane}

asdafdf

