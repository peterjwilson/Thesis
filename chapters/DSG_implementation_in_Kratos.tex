%%%%%%%%%%%%%%%%%%%%%%%%%%%%%%%%%%%%%%%%%%%%%%%%%%%%%%%%
%%%%                                              %%%%%%
%%%%  Author: Peter Wilson                        %%%%%%
%%%%                                              %%%%%%
%%%%  DSG element as programmed in Kratos                        %%%%%%
%%%%                                              %%%%%%
%%%%%%%%%%%%%%%%%%%%%%%%%%%%%%%%%%%%%%%%%%%%%%%%%%%%%%%%

\chapter[DSG element class in Kratos]{DSG element class in \\ Kratos}
\label{app:DSG element as programmed in Kratos}
\renewcommand{\Thema}{DSG element class in Kratos}

The DSG element is implemented in Kratos under the class \texttt{ShellThickElement3D3N} as per the established Kratos element naming convention. This appendix provides a brief description of all methods (arguments excluded for brevity) and variables within class \\ \texttt{ShellThickElement3D3N}.

\section{\texttt{ShellThickElement3D3N} methods}
\subsection{Public methods}
\texttt{ShellThickElement3D3N()}: Overloaded element constructor. Non-linear geometry bool has the default argument of \texttt{false}.

\texttt{\textasciitilde ShellThickElement3D3N()}: Element destructor.

\texttt{Create()}: Returns pointer to a new instance of this element.

\texttt{GetIntegrationMethod()}: Returns currently assigned integration method.

\texttt{Initialize()}: Initialises element with material properties, sets up isotropic or orthotropic cross section, initialises coordinate transformation and sets section orientation.

\texttt{ResetConstitutiveLaw()}: Resets the cross section by calling \\ \texttt{ShellCrossSection::ResetCrossSection()}. 

\texttt{EquationIdVector()}: Establishes correct relationship between global and local DOF numbering.

\texttt{GetDofList()}: Sets up vector of global numbering of local DOFs.

\texttt{Check()}: Ensures the validity of element data.

\texttt{GetValuesVector()}: Retrieves the element's nodal displacements from a specified historic timestep (default argument is last timestep).

\texttt{GetFirstDerivativesVector()}: Retrieves the element's nodal velocities from a specified historic timestep (default argument is last timestep).

\texttt{GetSecondDerivativesVector()}: Retrieves the element's nodal accelerations from a specified historic timestep (default argument is last timestep).

\texttt{InitializeNonLinearIteration()}: Accessibility function to access \\ \texttt{ShellCrossSection::InitializeNonLinearIteration()}.

\texttt{FinalizeNonLinearIteration()}: Accessibility function to access \\ \texttt{ShellCrossSection::FinalizeNonLinearIteration()}.

\texttt{InitializeSolutionStep()}: Accessibility function to access \\ \texttt{ShellCrossSection::InitializeSolutionStep()}. Initialises coordinate transformation for the current timestep.

\texttt{FinalizeSolutionStep()}: Accessibility function to access \\ \texttt{ShellCrossSection::FinalizeSolutionStep()}. Finalises coordinate transformation for the current timestep.

\texttt{CalculateMassMatrix()}: Calculates element mass matrix. Default is lumped.

\texttt{CalculateDampingMatrix()}: Placeholder for future implementation. Currently returns zero matrix.

\texttt{CalculateLocalSystem()}: Called by Kratos to add element stiffness contribution to global matrix. Internally re-directs to \texttt{CalculateAll()}.

\texttt{CalculateRightHandSide()}: Called by Kratos to add element internal force contribution to the global vector. Internally re-directs to \texttt{CalculateAll()}.

\texttt{GetValueOnIntegrationPoints()}: Overloaded method to calculate requested gauss point results.

\texttt{CalculateOnIntegrationPoints()}: Wrapper function callable with Python to access \\ \texttt{GetValueOnIntegrationPoints()}.

\texttt{Calculate()}: Overloaded method to calculate element specifics based on passed variable. Primarily used to access local element orientation and assign custom section rotations for composites without knowing the element's derived \texttt{element} type.

\texttt{SetCrossSectionsOnIntegrationPoints()}: Allows assigning custom cross sections to the element shell cross section.

\subsection{Private methods}
\texttt{CalculateStressesFromForceResultants()}: Calculates stresses from previously determined integrated force resultants as per reference \cite{FelippaKirchhoff2017}.

\texttt{CalculateLaminaStrains()}: Calculates the in-plane strains for the top and bottom surface of each lamina in the laminate as per equation \ref{eqscomp_strain_recovery2}.

\texttt{CalculateLaminaStresses()}: Calculates the in-plane stresses for the top and bottom surface of each lamina in the laminate as per equation \ref{eqscomp_stress_recovery1}.

\texttt{CalculateTsaiWuPlaneStress()}: Called for each lamina in the laminate. Determines the minimum Tsai-Wu reserve factor of the current lamina considering the top and bottom surfaces as per equation \ref{eqscomp_stress_recovery3}.

\texttt{CalculateVonMisesStress()}: Calculates the Von Mises equivalent stress as per equation \ref{eqt25}. Result can be calculated on the top, middle or bottom shell surface, or the maximum of all 3 surfaces depending on variable passed to method.

\texttt{CheckGeneralizedStressOrStrainOutput()}: Preliminary filter method called immediately by the \texttt{Matrix} overloaded \texttt{GetValueOnIntegrationPoints()} method to provide control flow.

\texttt{DecimalCorrection()}: Rounds very small entries of vectors to zero based on a dynamic numeric tolerance.

\texttt{SetupOrientationAngles()}: Allows the element cross section (material) to be rotated in-plane relative to the element geometry.

\texttt{CalculateSectionResponse()}: Calculates section response, including stress computation from availible strains, according to the constitutive law assigned. Also implements stabilization of shear constitutive entries as per equation \ref{eqt14}.

\texttt{InitializeCalculationData()}: Initialises element data invariant throughout the gauss loop necessary for the computation of the element stiffness matrix.

\texttt{CalculateDSGc3Contribution()}: Calculates the shear B-matrix as per the developmental DSGc3 formulation in equation \ref{eqDSGc3_26}.

\texttt{CalculateSmoothedDSGBMatrix()}: Calculates the shear B-matrix as per the CS-DSG formulation in equation \ref{eqCSDSG11}.

\texttt{CalculateDSGShearBMatrix()}: Called three times during CS-DSG element construction to calculate the sub-triangle's shear B-matrix.

\texttt{AddBodyForces()}: Adds self-weight forces based off the global acceleration specified in the main analysis.

\texttt{CalculateAll()}: The main pipeline to calculate the element stiffness matrix.

\texttt{TryGetValueOnIntegrationPoints\_MaterialOrientation()}: Provides a contour-representation of the element's material alignment.

\texttt{TryGetValueOnIntegrationPoints\_GeneralizedStrainsOrStresses()}: The main pipeline to calculate gauss point results such as strains, stresses, moments and integrated forces.

\texttt{printMatrix()}: Utility function to neatly print arbitrary matrices to the console for debugging.

\section{\texttt{ShellThickElement3D3N} member variables}
\subsection{Private}

class \texttt{CalculationData}: Contains the bulk of element data necessary for stiffness matrix computation. Detailed below.

\texttt{mpCoordinateTransformation}: Instance of the element's coordinate transformation. May be linear or non-linear corotational.

\texttt{mSections}: Container for the cross section instances associated with each integration point.

\texttt{mThisIntegrationMethod}: Current integration method of the element.

\texttt{mOrthotropicSectionRotation}: In-plane rotation angle for the element material cross section. Default is 0.0.

\subsection{Private, class \texttt{CalculationData} member variables}
All variables subsequently described are scoped to \texttt{CalculationData} unless noted otherwise.

\texttt{LCS0}: Reference coordinate system of undeformed configuration.

\texttt{LCS}: Current coordinate system of deformed configuration.

\texttt{dA}: Integration area associated with each gauss point. Equal to element area for a single gauss point.

\texttt{hMean}: Mean thickness of the shell cross section.

\texttt{TotalArea}: Total area of the element.

\texttt{gpLocations}: Vector of gauss point locations, used for results output.

\texttt{dNxy}: Cartesian derivatives of shape functions.

\texttt{N}: Shape function vector at the current integration point.

\texttt{globalDisplacements}: Global displacement vector.

\texttt{localDisplacements}: Local displacement vector.

\texttt{CalculateRHS}: Bool to calculate internal force vector.

\texttt{CalculateLHS}: Bool to calculate stiffness matrix.

\texttt{parabolic\_composite\_transverse\_shear\_strains}: Bool to calculate composite transverse shear strains according to a constant (equation \ref{eqscomp_strain_recovery3}) or parabolic (equation \ref{eqscomp_strain_recovery6}) profile. Default value is \texttt{false}.

\texttt{basicTriCST}: Bool to use a basic displacement-derived shear formulation as per equation \ref{eqApp2_3}. Default value is \texttt{false}.

\texttt{ignore\_shear\_stabilization}: Bool to ignore constitutive shear matrix stabilization as per equation \ref{eqt14}. Default value is \texttt{false}.

\texttt{smoothedDSG}: Bool to use the CS-DSG formulation as per \ref{eqCSDSG11}. Default value is \texttt{false}.

\texttt{specialDSGc3}: Bool to use the DSGc3 formulation as per \ref{eqDSGc3_26}. Default value is \texttt{false}.

\texttt{gpIndex}: Integer holding the index of the current gauss point.

\texttt{B}: Element strain-displacement B matrix.

\texttt{h\_e}: Length of the element's longest edge, forming part of the constitutive matrix shear stabilization calculation in equation \ref{eqt14}.

\texttt{alpha}: Constant-value alpha parameter forming part of the constitutive matrix shear stabilization calculation in equation \ref{eqt14}.

\texttt{shearStabilisation}: Condensation of all terms modifying the shear constitutive matrix as per equation \ref{eqt14}.

\texttt{D}: Element material matrix.

\texttt{generalizedStrains}: Element strain vector.

\texttt{generalizedStresses}: Element stress vector.

\texttt{SectionParameters}: Parameters of the shell cross section calculation.

\texttt{rlaminateStrains}: Vector of element lamina strain vectors.

\texttt{rlaminateStresses}: Vector of element lamina stress vectors.

\texttt{CurrentProcessInfo}: Reference to the Kratos job's current process information.