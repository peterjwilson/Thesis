%%%%%%%%%%%%%%%%%%%%%%%%%%%%%%%%%%%%%%%%%%%%%%%%%%%%%%%%
%%%%                                              %%%%%%
%%%%  Author: Peter Wilson                        %%%%%%
%%%%                                              %%%%%%
%%%%  Background of shells                        %%%%%%
%%%%                                              %%%%%%
%%%%%%%%%%%%%%%%%%%%%%%%%%%%%%%%%%%%%%%%%%%%%%%%%%%%%%%%

\chapter{Basic-DKQ formulation}
\label{sec:Basic-DKQ quadrilateral formulation}
\renewcommand{\Thema}{Basic-DKQ quadrilateral formulation}

Introduced in section \ref{section:shell obstacle course}, the Basic-DKQ quadrilateral element represents the ANDES-DKQ formulation with it's membrane ANDES element technology replaced with an un-enhanced displacement based membrane formulation. The DKQ bending formulation is used as per section \ref{section:DKQ bending formulation}.

The membrane stiffness of the Basic-DKQ formulation is purely displacement based and employs the standard bi-linear quadrilateral shape functions:
\begin{gather} 
\begin{aligned}
&N_1 (\xi , \eta) = \frac{1}{4} (1-\xi)(1-\eta)\ ,
\\
&N_2 (\xi , \eta) = \frac{1}{4} (1+\xi)(1-\eta)\ ,
\\
&N_3 (\xi , \eta) = \frac{1}{4} (1+\xi)(1+\eta)\ and
\\
&N_4 (\xi , \eta) = \frac{1}{4} (1-\xi)(1+\eta)
\label{eqApp1_1}\ .
\end{aligned}
\end{gather}
The membrane strains are related to the discrete in-plane membrane displacements $\hat{\mathbf{v}}$ via the differential operator $\mathbf{L}$ and the aforementioned shape functions $\mathbf{N}$ as such:
\begin{equation}
\boldsymbol{\epsilon} = \mathbf{L} \mathbf{N} \hat{\mathbf{v}} = \mathbf{B}  \hat{\mathbf{v}} 
\hspace{5mm}
with
\hspace{5mm}
\hat{\mathbf{v}} = \begin{pmatrix}
\hat{\mathbf{v}}_1 \\
\hat{\mathbf{v}}_2 \\
\hat{\mathbf{v}}_3 \\
\hat{\mathbf{v}}_4 \\
\end{pmatrix}
\hspace{5mm}
and
\hspace{5mm}
\hat{\mathbf{v}}_i = \begin{pmatrix}
\hat{{v}}_{xi} \\
\hat{{v}}_{yi}
\end{pmatrix}
\label{eqApp1_2}\ .
\end{equation}
The differential operator is defined as:
\begin{equation}
\mathbf{L} = \begin{pmatrix}
\frac{\partial}{\partial x} & 0 \\
0 & \frac{\partial}{\partial y}\\
\frac{\partial}{\partial y} & \frac{\partial}{\partial x}
\end{pmatrix}
\label{eqApp1_3}\ .
\end{equation}

The matrix of shape functions is structured as follows:
\begin{equation}
\mathbf{N}=  \begin{pmatrix}
\mathbf{N}_1 & \mathbf{N}_2 & \mathbf{N}_3 & \mathbf{N}_4
\end{pmatrix}
\hspace{5mm}
with
\hspace{5mm}
\mathbf{N}_i=  \begin{pmatrix}
{N}_i & 0\\
0& N_i
\end{pmatrix}
\label{eqApp1_4}\ .
\end{equation}
The basic membrane strain displacement matrix $\mathbf{B}$ is thus:
\begin{equation}
\mathbf{B}=  \mathbf{L} \mathbf{N} = \begin{pmatrix}
\mathbf{B}_1 & \mathbf{B}_2 & \mathbf{B}_3 & \mathbf{B}_4
\end{pmatrix}
\hspace{5mm}
with
\hspace{5mm}
\mathbf{B}_i=  \begin{pmatrix}
{N}_{i,x} & 0\\
0& {N}_{i,y}\\
{N}_{i,y} & {N}_{i,x}
\end{pmatrix}
\label{eqApp1_5}\ .
\end{equation}
The entries of which can be calculated with the help of the Jacobian $\mathbf{J}$:
\begin{equation}
\mathbf{J}=  \begin{pmatrix}
\frac{\partial x}{\partial \xi} & \frac{\partial y}{\partial \xi} \\
\frac{\partial x}{\partial \eta} & \frac{\partial y}{\partial \eta}
\end{pmatrix}
\label{eqApp1_6}\ ,
\end{equation}
applied to the shape function parametric derivatives yields
\begin{equation}
\mathbf{J}^{-1} 
\begin{pmatrix}
{N}_{1,\xi} & {N}_{2,\xi} & {N}_{3,\xi} & {N}_{4,\xi}\\
{N}_{1,\eta} & {N}_{2,\eta} & {N}_{3,\eta} & {N}_{4,\eta}
\end{pmatrix}
=
\begin{pmatrix}
{N}_{1,x} & {N}_{2,x} & {N}_{3,x} & {N}_{4,x}\\
{N}_{1,y} & {N}_{2,y} & {N}_{3,y} & {N}_{4,y}
\end{pmatrix}
\label{eqApp1_7}\ .
\end{equation}

The above strain displacement matrix doesn't cover drilling stiffnesses, so an artificial drilling stiffness as per the DSG element formulation equation \ref{eqtdrilling} was added for each node after the construction of the element stiffness matrix:

\begin{equation} 
K_{\beta_z} =  \frac{max(K_{ij}\delta_{ij})}{1000}
\label{eqtdrilling1}\ .
\end{equation}