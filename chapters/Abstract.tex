%%%%%%%%%%%%%%%%%%%%%%%%%%%%%%%%%%%%%%%%%%%%%%%%%%%%%%%%
%%%%                                              %%%%%%
%%%%  Author: Name des Autors                     %%%%%%
%%%%                                              %%%%%%
%%%%  Beschreibung:                               %%%%%%
%%%%                                              %%%%%%
%%%%%%%%%%%%%%%%%%%%%%%%%%%%%%%%%%%%%%%%%%%%%%%%%%%%%%%%

\chapter*{Abstract}
\label{cha:abstract}
\lettrine[lines=2]{A}{s} the use of Finite Element Methods (FEMs) proliferate throughout both academia and industry so does the need to curb ill-conceived shell finite element analyses. Exacerbated by the prevalence of commercial "black box" codes, the ease with which verisimilitudinous results can be obtained poses a unique risk compared to classical engineering methods. Advanced shell finite elements with enhancing element technologies, such as the linear triangle DSG and the linear quadrilateral ANDES-DKQ elements implemented and validated in this thesis, have proven themselves robust enough for general purpose analysis and no doubt aid in tempering the risk of incorrect analysis. However, simply employing advanced shell finite elements does not automatically inoculate against spurious modelling. Correct understanding of shell theories and the shell finite elements themselves gives rise to the correct structural modelling of shell finite elements, a detailed study of which is presented in this work. Consolidation of advanced shell finite elements and their proper structural modelling effectively mitigates this risk, resulting in confident and accurate analyses.

\vspace*{10mm}

%\section*{Keywords}
{\textcolor{gray75}{\Huge\bfseries{Keywords}}}

\vspace*{8mm}

\keywords
%{\textcolor{gray75}{\Huge\bfseries{Keywords}}}

\newpage