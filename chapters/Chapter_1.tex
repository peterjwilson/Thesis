%%%%%%%%%%%%%%%%%%%%%%%%%%%%%%%%%%%%%%%%%%%%%%%%%%%%%%%%
%%%%                                              %%%%%%
%%%%  Author: Peter Wilson                        %%%%%%
%%%%                                              %%%%%%
%%%%  Introduction                                %%%%%%
%%%%                                              %%%%%%
%%%%%%%%%%%%%%%%%%%%%%%%%%%%%%%%%%%%%%%%%%%%%%%%%%%%%%%%


%fref generates automatically the respective abreviation/word in the text for the reference. You just have to define a label starting with the respective keyword.
%english: chap, sec, fig, eq, app
%deutsch: chap/kap, abs, abb, gl, anh
%see http://ctan.space-pro.be/tex-archive/macros/latex/contrib/fancyref/fancyref.pdf for more information

\chapter{Introduction}
\label{chap:chapter_1}

\renewcommand{\Thema}{Introduction}

\lettrine[lines=2]{D}{riving} the increasing adoption of the Finite Element Method (FEM) in both academia and the industry are a myriad of competing demands such as greater strength while simultaneously achieving leaner designs and increased analysis accuracy at lower computational costs. Many engineering scenarios previously analysed with classical hand calculations are now finalized or replaced with the use of Finite Element Analysis (FEA), or, indeed, the aforementioned pressures drive designs into new realms that fall outside the purview of hand calculations entirely. Detailed analysis of conventional structural steelwork and innovative design of unconventional lightweight shell structures are but two examples of the widespread embrace of FEA, both of which commonly employ shell finite elements to accurately resolve structural behaviour. Given the availability of "black box" commercial FEM codes and the ease with which ostensibly convincing shell models can be created, a general lack of shell theory knowledge manifests a void subsequently filled with questionable results. Conversely, shell theory knowledge allows one to appreciate both the critical behaviour of the structure and also realise the limitations of various shell finite elements, the reconciliation of these two items in conjunction with advanced robust shell finite elements culminates in confident and accurate analyses.

The objective of this work is two-fold:
\begin{enumerate}
	\item Implement two advanced shell finite elements in the multi-physics code Kratos with the following functionality:
	\begin{itemize}
		\item isotropic and orthotropic laminate linear elastic materials,
		\item geometrically linear and non-linear analysis,
		\item static and dynamic analysis and
		\item a wide range of quantity recovery options.\\
	\end{itemize}

	\item Illuminate the structural modelling of advanced shell finite elements by examining the interaction between structural behaviour, base formulations, enhancing technologies and formulation-mesh-dependency.
\end{enumerate}


\newpage
This thesis can be divided into three parts:
\begin{itemize}
	\item \textbf{Part 1: Background theory}
	
	Chapters 2 - 4 cover the relevant theory pertinent not only to the implementation of the shells in Kratos, but also to the theoretical understanding necessary for an informed discussion of shell structural modelling.
	\begin{itemize}
		\item Chapter 2 provides an overview of common mathematical shell models and their associated assumptions and limitations. Artificial locking effects that arise from the translation of these mathematical models into low order finite elements are discussed, as well as various element technologies proposed as remedies. From this, the base formulations and enhancing technologies of the Kratos elements to be implemented are chosen.
		\item Chapter 3 establishes composite material basics and common composite nomenclature. The internal work of a 5-parameter orthotropic laminate shell is developed, leading to expressions for the integrated laminate constitutive matrix and integrated force resultants. Laminae stress and strain recovery are subsequently covered, followed by the Tsai-Wu failure criterion.
		\item Chapter 4 covers a general overview of non-linear analysis. Response diagrams and critical points are explored through the lens of stability analysis, while an outline of the co-rotational approach and the element independent co-rotational approach, employed in Kratos, are subsequently offered.
	\end{itemize}


	\item \textbf{Part 2: Implementation of shell finite elements in Kratos}
	
	Chapters 5 - 8 primarily deal with the implementation of the advanced shell finite elements in Kratos and their validation.
	\begin{itemize}
		\item Chapter 5 walks through the DSG linear triangle shell element formulation and implementation in Kratos. The stiffness matrix formulation and implementation, lumped and consistent mass matrix details and stress and strain recovery are presented.
		\item Chapter 6 goes through the ANDES-DKQ linear quadrilateral shell element formulation and implementation in Kratos, surveying the same points as chapter 5.
		\item Chapter 7 extends both elements from isotropic materials to orthotropic composite laminates by covering the relevant constitutive matrices, stress and strain recovery and Tsai-Wu failure criterion details.
		\item Chapter 8 demonstrates the correct implementation and accuracy of the elements with validation tests spanning linear statics, non-linear statics, linear dynamics and non-linear dynamics across isotropic and orthotropic composite materials. Recovery of stresses, strains, integrated forces, Von Mises stresses and the composite Tsai-Wu reserve index are also validated.
	\end{itemize}
	\newpage
	\item \textbf{Part 3: Finite element structural modelling}
	
	Chapters 9 and 10 examine the structural modelling of finite shell elements, interrogating the interplay between structural behaviour, base formulations, enhancing technologies and formulation-mesh-dependency.
	\begin{itemize}
		\item Chapter 9 considers the detailed investigation of two geometrically non-linear example problems: Euler beam buckling and the shear wrinkling of a flat plate. For each case the structural behaviour is compared across elements of different base formulations, with enhancing technologies switched on and off to further extricate the underlying phenomena either properly or improperly resolved by the various shell structural models.
		\item Chapter 10 looks at the extension of DSG linear triangle element technology into a formulation invariant of nodal numbering. A developmental proof of concept is considered, followed by a published DSG extension formulation whose behaviour does not depend on nodal ordering. Tying back to structural modelling, an appraisal of the DSG formulations is put forth with the aim of recommending the preferred DSG linear triangle element for general analysis.
	\end{itemize}
\end{itemize}


The programming work associated with this thesis was completed in Kratos (links below), a multi-physics code with a plethora of individual applications including Structural Mechanics, Fluid Dynamics, Fluid Structure Interaction, Discrete Element Modelling and Shape Optimization, many of which can be combined seamlessly into multi-disciplinary analyses. Emerging from the International Center for Numerical Methods in Engineering (CIMNE) in Barcelona and co-developed by the Technical University of Munich (TUM), Kratos's applications are primarily written in C++, while Python is also utilised for efficient communication between applications and with the user.

\vspace*{10mm}

\begin{figure}[H]
	\centering
	\def\svgwidth{\columnwidth}
	\includegraphics[width=10cm]{images/kratoslogo.png}
	\label{kratoslogo}
\end{figure}

\begin{center}
CIMNE Kratos Multi-physics homepage: \\
\textit{http://www.cimne.com/kratos/}

Kratos Multi-physics Github: \\
\textit{https://github.com/KratosMultiphysics}

Kratos Multi-physics Github wiki, application cases: \\
 \textit{https://github.com/KratosMultiphysics/Kratos/wiki/Application-Cases}
\end{center}