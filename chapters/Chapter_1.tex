%%%%%%%%%%%%%%%%%%%%%%%%%%%%%%%%%%%%%%%%%%%%%%%%%%%%%%%%
%%%%                                              %%%%%%
%%%%  Author: Peter Wilson                        %%%%%%
%%%%                                              %%%%%%
%%%%  Introduction                                %%%%%%
%%%%                                              %%%%%%
%%%%%%%%%%%%%%%%%%%%%%%%%%%%%%%%%%%%%%%%%%%%%%%%%%%%%%%%


%fref generates automatically the respective abreviation/word in the text for the reference. You just have to define a label starting with the respective keyword.
%english: chap, sec, fig, eq, app
%deutsch: chap/kap, abs, abb, gl, anh
%see http://ctan.space-pro.be/tex-archive/macros/latex/contrib/fancyref/fancyref.pdf for more information

\chapter{Introduction}
\label{chap:chapter_1}

\renewcommand{\Thema}{Introduction}

KRATOS is a relatively recent multiphysics code emerging from the International Center for Numerical Methods in Engineering (CIMNE) in Barcelona. Primarily aimed at developers, researchers and students, KRATOS's extensibility accommodates the introduction of new functionalities with relative ease.

The additional functionality considered in this paper is the implementation of a thin quadrilateral shell element. This new contribution fills an existing gap in the structural mechanics capabilities of KRATOS, which is currently missing thin quadrilateral and thick triangular shell elements.

Shell elements themselves result from the combination of membrane and bending behaviours into a single element. The thin quadrilateral shell element presented in this paper consists of an Assumed Natural Deviatoric Strains (ANDES) membrane formulation and a complementing Discrete Kirchhoff Quadrilateral (DKQ) bending formulation. Both of these component formulations are discussed in section 1 of the paper. Following this, a high level overview of the element's implementation in KRATOS is presented in section 2, while section 3 covers element benchmarking with the well known shell obstacle course. Future work is discussed in section 4 of this paper.