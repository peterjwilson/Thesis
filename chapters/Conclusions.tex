%%%%%%%%%%%%%%%%%%%%%%%%%%%%%%%%%%%%%%%%%%%%%%%%%%%%%%%%
%%%%                                              %%%%%%
%%%%  Author: Name des Autors                     %%%%%%
%%%%                                              %%%%%%
%%%%  Beschreibung:                               %%%%%%
%%%%                                              %%%%%%
%%%%%%%%%%%%%%%%%%%%%%%%%%%%%%%%%%%%%%%%%%%%%%%%%%%%%%%%

\chapter{Conclusions and Outlook - STILL TO DO!!!!!}
\label{chap:conclusions}
\renewcommand{\Thema}{Conclusion}

This work has considered the implementation of a thin quadrilateral shell element for the multiphysics code KRATOS. Section 1 covered the shell formulation, which is split into membrane and bending components. Following this, the element's implementation in KRATOS was presented, which covered key methods employed and the general workflow to calculate the element stiffness matrix. The element considered was subjected to the well known shell obstacle course in Section 3. Although the element correctly converged to the reference solution for the Scordelis-Lo roof and Pinched Cylinder problem, the Pinched Hemisphere benchmark revealed element deficiencies. These deficiencies were identified, with informed direction suggested  for future work associated with improving this element. 

\section{Future work}

Extend shell elements model into a variety of directions:

- Consider von karman non-linear strains for very thin shells

- Improve transverse shear stress modelling for composites
Look at paper: Improved Transverse Shear Stresses in Composite Finite Elements based on First Order Shear Deformation Theory
Or, develop separate element with higher order resolving theory

- Extend the DSG element into XFEM as per \cite{DSG_XFEM_2015}

- Continue development of the DSGc3 formulation